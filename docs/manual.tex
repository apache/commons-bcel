\documentclass[12pt,twoside]{article}

\usepackage{epsf,a4wide,moreverb,url}
\usepackage{palatino}

\newcommand\jc{{\sffamily BCEL }}
\newcommand\cp{{constant pool }}
\newcommand\cpe{constant pool}
\newcommand\jvm{{Java Virtual Machine }}
\newcommand\jvme{{Java Virtual Machine}}
\newcommand\vm{{Virtual Machine }}
\newcommand\href[2]{#2}

\begin{document}

\title{Byte Code Engineering Library (BCEL)\\
       Description and usage manual\\
       {\small \textbf{Version 1.0}}}

\author{{\Large Markus Dahm}\\\\
        \href{mailto:markus.dahm@inf.fu-berlin.de}{\texttt{markus.dahm@berlin.de}}}

\maketitle

%\tableofcontents

\begin{abstract}
Extensions  and improvements of the  programming language Java and its
related  execution environment (Java   Virtual Machine,  JVM)  are the
subject  of a large number  of research  projects and proposals. There
are  projects, for  instance, to add   parameterized types to Java, to
implement ``Aspect-Oriented Programming'', to perform sophisticated
static analysis, and to improve the run-time performance.

Since  Java classes  are  compiled into  portable  binary class  files
(called   \emph{byte   code}),  it   is   the   most  convenient   and
platform-independent  way  to  implement  these  improvements  not  by
writing a  new compiler or changing  the JVM, but  by transforming the
byte  code.   These  transformations  can either  be  performed  after
compile-time,  or at load-time.   Many programmers  are doing  this by
implementing their own specialized byte code manipulation tools, which
are, however, restricted in the range of their re-usability.

To deal with the necessary class file transformations, we introduce an
API   that   helps   developers   to  conveniently   implement   their
transformations.
\end{abstract}

\section{Introduction}\label{sec:intro}

The  Java language  \cite{gosling} has  become very  popular  and many
research projects  deal with further  improvements of the  language or
its run-time behavior.  The possibility  to extend a language with new
concepts  is surely  a  desirable feature,  but implementation  issues
should be hidden from the user.  Fortunately, the concepts of the \jvm
permit  the user-transparent  implementation of  such  extensions with
relatively little effort.

Because the target language of  Java is an interpreted language with a
small  and  easy-to-understand  set  of instructions  (the  \emph{byte
code}), developers  can implement  and test their  concepts in  a very
elegant way.   One can  write a plug-in  replacement for  the system's
class loader which is  responsible for dynamically loading class files
at  run-time  and  passing the  byte  code  to  the \vm  (see  section
\ref{sec:classloaders}).  Class loaders may  thus be used to intercept
the  loading process and  transform classes  before they  get actually
executed  by the  JVM  \cite{classloader}.  While  the original  class
files always remain unaltered, the behavior of the class loader may be
reconfigured for every execution or instrumented dynamically.

The \jc API (Byte Code Engineering Library), formerly known as
JavaClass, is a toolkit for the static analysis and dynamic creation
or transformation of Java class files.  It enables developers to
implement the desired features on a high level of abstraction without
handling all the internal details of the Java class file format and
thus re-inventing the wheel every time.  \jc is written entirely in
Java and freely available under the terms of the Apache Software
License.  \footnote{The distribution is available at
  \url{http://jakarta.apache.org/bcel/}, including several code
  examples and javadoc manuals.  }

This paper is structured as follows:  We give a  brief description of
the \jvm and the class  file format in section \ref{sec:jvm}.  Section
\ref{sec:api} introduces  the \jc API.   Section \ref{sec:application}
describes  some typical  application areas  and example  projects. The
appendix contains  code examples that are  to long to  be presented in
the  main  part  of this  paper.  All  examples  are included  in  the
down-loadable distribution.

\subsection{Related work}

There are  a number  of proposals and  class libraries that  have some
similarities with \textsc{BCEL}: The JOIE \cite{joie} toolkit can
be used to instrument class loaders with dynamic behavior.  Similarly,
``Binary  Component Adaptation''  \cite{bca} allows  components  to be
adapted and  evolved on-the-fly.  Han  Lee's ``Byte-code Instrumenting
Tool'' \cite{bit} allows the user  to insert calls to analysis methods
anywhere in the  byte code.  The Jasmin language  \cite{jasmin} can be
used  to   hand-write  or  generate   pseudo-assembler  code.   D-Java
\cite{classfile} and JCF \cite{inside} are class viewing tools.

In contrast to these projects, \jc is intended to be a general purpose
tool  for ``byte  code engineering''.   It gives  full control  to the
developer on a high level of  abstraction and is not restricted to any
particular application area.

\section{The Java Virtual Machine}\label{sec:jvm}

Readers already familiar with the \jvm and the Java class file format
may want to skip this section and proceed with section \ref{sec:api}.

Programs written  in the  Java language are  compiled into  a portable
binary format called \emph{byte  code}.  Every class is represented by
a  single class  file  containing  class related  data  and byte  code
instructions. These  files are loaded dynamically  into an interpreter
(\jvme, JVM) and executed.

Figure  \ref{fig:jvm}  illustrates  the  procedure  of  compiling  and
executing a Java class:  The source file (\texttt{HelloWorld.java}) is
compiled into a Java class file (\texttt{HelloWorld.class}), loaded by
the  byte  code  interpreter  and  executed.  In  order  to  implement
additional  features, researchers  may want  to transform  class files
(drawn  with  bold lines)  before  they  get  actually executed.  This
application area is one of the main issues of this article.

\begin{figure}[htbp]
  \begin{center}
    \leavevmode
    \epsfxsize\textwidth
    \epsfbox{eps/jvm.eps}
    \caption{Compilation and execution of Java classes}
    \label{fig:jvm}
  \end{center}
\end{figure}

Note that the  use of the general term  ``Java'' implies two meanings:
on the one hand, Java as a programming language is meant, on the other
hand, the Java  Virtual Machine, which is not  necessarily targeted by
the Java language  exclusively, but may be used  by other languages as
well (e.g.   Eiffel \cite{eiffel}, or Ada \cite{ada}).   We assume the
reader to  be familiar with  the Java language  and to have  a general
understanding of the Virtual Machine.

\subsection{Java class file format}\label{sec:format}

Giving a  full overview of  the design issues  of the Java  class file
format and the  associated byte code instructions is  beyond the scope
of this paper.   We will just give a  brief introduction covering the
details  that  are  necessary  for  understanding  the  rest  of  this
paper. The format of class files and the byte code instruction set are
described  in more  detail  in the  ``\jvm Specification''  \cite{jvm}
\footnote{Also             available             online             at
\url{http://www.javasoft.com/docs/books/vmspec/index.html}},   and  in
\cite{jasmin}.   Especially,  we  will  not  deal  with  the  security
constraints that the \jvm has to check at run-time, i.e. the byte code
verifier.

Figure \ref{fig:classfile} shows a  simplified example of the contents
of a  Java class file:  It starts with  a header containing  a ``magic
number'' (\texttt{0xCAFEBABE}) and the version number, followed by the
\emph{\cpe}, which can be roughly thought of as the text segment of an
executable, the  \emph{access rights}  of the class  encoded by  a bit
mask, a list of interfaces  implemented by the class, lists containing
the  fields and  methods of  the  class, and  finally the  \emph{class
attributes}, e.g.  the  \texttt{SourceFile} attribute telling the name
of  the source  file.  Attributes  are  a way  of putting  additional,
e.g. user-defined,  information into class file  data structures.  For
example, a  custom class  loader may evaluate  such attribute  data in
order to perform its  transformations.  The JVM specification declares
that unknown, i.e.  user-defined attributes must be ignored by any \vm
implementation.

\begin{figure}[htbp]
  \begin{center}
    \leavevmode
    \epsfxsize\textwidth
    \epsfbox{eps/classfile.eps}
    \caption{Java class file format}
    \label{fig:classfile}
  \end{center}
\end{figure}

Because  all of  the  information needed  to  dynamically resolve  the
symbolic  references to  classes, fields  and methods  at  run-time is
coded  with string  constants, the  \cp contains  in fact  the largest
portion of an average class file, approximately 60\% \cite{statistic}.
The byte code instructions themselves just make up 12\%.

The right upper box shows a  ``zoomed'' excerpt of the \cpe, while the
rounded box below depicts  some instructions that are contained within
a  method  of the  example  class.  These  instructions represent  the
straightforward translation of the well-known statement:

\begin{verbatim}
   System.out.println("Hello, world");
\end{verbatim}

The first instruction loads the  contents of the field \texttt{out} of
class  \texttt{java.lang.System} onto  the operand  stack. This  is an
instance of  the class \texttt{java.io.PrintStream}.  The \texttt{ldc}
(``Load constant'') pushes a reference  to the string "Hello world" on
the  stack.    The  next  instruction  invokes   the  instance  method
\texttt{println}  which  takes  both  values as  parameters  (Instance
methods always  implicitly take an  instance reference as  their first
argument).

Instructions,  other  data  structures   within  the  class  file  and
constants  themselves  may  refer  to  constants  in  the  \cpe.  Such
references are implemented via fixed indexes encoded directly into the
instructions.   This  is illustrated  for  some  items  of the  figure
emphasized    with   a    surrounding   box.

For  example,  the  \texttt{invokevirtual}  instruction  refers  to  a
\texttt{MethodRef} constant  that contains information  about the name
of the  called method, the  signature (i.e.  the encoded  argument and
return types),  and to  which class the  method belongs.  In  fact, as
emphasized by the boxed  value, the \texttt{MethodRef} constant itself
just refers to other entries holding the real data, e.g.  it refers to
a \texttt{ConstantClass} entry containing  a symbolic reference to the
class \texttt{java.io.PrintStream}.   To keep the  class file compact,
such  constants  are   typically  shared  by  different  instructions.
Similarly, a field is represented by a \texttt{Fieldref} constant that
includes information about the name, the type and the containing class
of the field.

The \cp  basically holds the following types  of constants: References
to methods, fields and  classes, strings, integers, floats, longs, and
doubles.

\subsection{Byte code instruction set}\label{sec:code}

The JVM  is a  stack-oriented interpreter that  creates a  local stack
frame of fixed size for every method invocation. The size of the local
stack has to  be computed by the compiler.  Values  may also be stored
intermediately in a frame area containing \emph{local variables} which
can  be used  like  a set  of  registers.  These  local variables  are
numbered from 0  to 65535, i.e.  you have a maximum  of 65536 of local
variables.   The  stack  frames   of  caller  and  callee  method  are
overlapping, i.e.  the caller  pushes arguments onto the operand stack
and the called method receives them in local variables.

The byte code instruction  set currently consists of 212 instructions,
44  opcodes  are  marked  as  reserved  and may  be  used  for  future
extensions   or   intermediate   optimizations  within   the   Virtual
Machine. The instruction set can be roughly grouped as follows:

\begin{description}
\item[Stack operations:] Constants can be pushed onto the stack either
by loading them from the \cp with the \texttt{ldc} instruction or with
special ``short-cut''  instructions where the operand  is encoded into
the  instructions, e.g.   \texttt{iconst\_0} or  \texttt{bipush} (push
byte value).

\item[Arithmetic  operations:]   The  instruction  set   of  the  \jvm
distinguishes  its  operand  types  using  different  instructions  to
operate on  values of  specific type.  Arithmetic  operations starting
with  \texttt{i}, for  example,  denote  an  integer  operation.  E.g.,
\texttt{iadd} that adds two integers and pushes the result back on the
stack.     The    Java    types    \texttt{boolean},    \texttt{byte},
\texttt{short}, and \texttt{char} are handled as integers by the JVM.

\item[Control flow:] There  are branch instructions like \texttt{goto}
and   \texttt{if\_icmpeq},    which   compares   two    integers   for
equality.  There  is  also   a  \texttt{jsr}  (jump  sub-routine)  and
\texttt{ret} pair of instructions that  is  used to implement the
\texttt{finally} clause of  \texttt{try-catch} blocks.  Exceptions may
be thrown with the \texttt{athrow} instruction.

Branch  targets  are coded  as  offsets  from  the current  byte  code
position, i.e. with an integer number.

\item[Load   and   store operations]    for   local   variables   like
\texttt{iload} and  \texttt{istore}.  There are also  array operations
like \texttt{iastore} which stores an integer value into an array.

\item[Field access:] The  value of an instance field  may be retrieved
with \texttt{getfield} and written with \texttt{putfield}.  For static
fields,   there    are   \texttt{getstatic}   and   \texttt{putstatic}
counterparts.

\item[Method  invocation:] Methods  may  either be  called via  static
references with \texttt{invokesta\-tic} or be bound virtually with the
\texttt{invokevirtual}  instruction. Super  class methods  and private
methods are invoked with \texttt{invokespecial}.

\item[Object  allocation:]  Class  instances  are allocated  with  the
\texttt{new}  instruction, arrays  of basic  type  like \texttt{int[]}
with \texttt{newarray}, arrays  of references like \texttt{String[][]}
with \texttt{anewarray} or \texttt{multianewarray}.

\item[Conversion and type checking:]  For stack operands of basic type
there  exist casting  operations  like \texttt{f2i}  which converts  a
float  value into  an integer.   The validity  of a  type cast  may be
checked with  \texttt{checkcast} and the  \texttt{instanceof} operator
can be directly mapped to the equally named instruction.
\end{description}

Most  instructions  have a  fixed  length,  but  there are  also  some
variable-length instructions: In particular, the \texttt{lookupswitch}
and  \texttt{tableswitch} instructions,  which are  used  to implement
\texttt{switch()}  statements.   Since  the  number  of  \texttt{case}
clauses  may vary,  these instructions  contain a  variable  number of
statements.

We  will not list  all byte  code instructions  here, since  these are
explained in  detail in the  JVM specification.  The opcode  names are
mostly self-explaining,  so understanding the  following code examples
should be fairly intuitive.

\subsection{Method code}\label{sec:code2}

Non-abstract methods  contain an attribute  (\texttt{Code}) that holds
the following data: The maximum  size of the method's stack frame, the
number   of   local   variables    and   an   array   of   byte   code
instructions. Optionally,  it may  also contain information  about the
names of local variables and source file line numbers that can be used
by a debugger.

Whenever  an exception is thrown, the  JVM performs exception handling
by looking   into a  table  of exception  handlers.   The table  marks
handlers, i.e.  pieces  of code, to  be responsible for  exceptions of
certain types  that  are raised   within a  given  area  of  the  byte
code. When there is no appropriate handler the exception is propagated
back to the caller of the method. The handler information is itself
stored in an attribute contained within the \texttt{Code} attribute.

\subsection{Byte code offsets}\label{sec:offsets}

Targets  of  branch instructions  like  \texttt{goto}  are encoded  as
relative offsets  in the array  of byte codes. Exception  handlers and
local variables refer to absolute addresses within the byte code.  The
former  contains  references   to  the  start  and  the   end  of  the
\texttt{try} block,  and to the instruction handler  code.  The latter
marks the  range in which a  local variable is valid,  i.e. its scope.
This makes it  difficult to insert or delete code  areas on this level
of abstraction, since one has  to recompute the offsets every time and
update the referring objects. We will see in section \ref{sec:cgapi}
how \jc remedies this restriction.

\subsection{Type information}\label{sec:types}

Java is  a type-safe language and  the information about  the types of
fields,    local    variables,    and    methods    is    stored    in
\emph{signatures}. These are strings stored  in the \cp and encoded in
a special  format.  For example the  argument and return  types of the
\texttt{main} method

\begin{verbatim}
  public static void main(String[] argv)
\end{verbatim}

are represented by the signature

\begin{verbatim}
  ([java/lang/String;)V
\end{verbatim}

Classes  and  arrays  are   internally  represented  by  strings  like
\texttt{"java/lang/String"},  basic types  like  \texttt{float} by  an
integer number. Within signatures they are represented by single
characters, e.g., \texttt{"I"}, for integer.

\subsection{Code example}\label{sec:fac}

The  following example  program prompts  for a  number and  prints the
faculty  of  it.  The  \texttt{readLine()}  method  reading  from  the
standard input  may raise an \texttt{IOException} and  if a misspelled
number    is    passed   to    \texttt{parseInt()}    it   throws    a
\texttt{NumberFormatException}. Thus, the critical area of code must be
encapsulated in a \texttt{try-catch} block.

{\small \begin{verbatim}
import java.io.*;
public class Faculty {
  private static BufferedReader in = new BufferedReader(new
                                InputStreamReader(System.in));
  public static final int fac(int n) {
    return (n == 0)? 1 : n * fac(n - 1);
  }
  public static final int readInt() {
    int n = 4711;
    try {
      System.out.print("Please enter a number> ");
      n = Integer.parseInt(in.readLine());
    } catch(IOException e1) { System.err.println(e1); }
      catch(NumberFormatException e2) { System.err.println(e2); }
    return n;
  }
  public static void main(String[] argv) {
    int n = readInt();
    System.out.println("Faculty of " + n + " is " + fac(n));
  }}
\end{verbatim}}

This code example  typically compiles to the following  chunks of byte
code:

\subsubsection{Method fac}

{\small \begin{verbatim}
0:  iload_0
1:  ifne            #8
4:  iconst_1
5:  goto            #16
8:  iload_0
9:  iload_0
10: iconst_1
11: isub
12: invokestatic    Faculty.fac (I)I (12)
15: imul
16: ireturn

LocalVariable(start_pc = 0, length = 16, index = 0:int n)
\end{verbatim}}

The  method \texttt{fac}  has only  one local  variable,  the argument
\texttt{n}, stored in  slot 0.  This variable's scope  ranges from the
start of  the byte  code sequence to  the very  end.  If the  value of
\texttt{n} (stored  in local variable  0, i.e. the value  fetched with
\texttt{iload\_0}) is  not equal  to 0, the  \texttt{ifne} instruction
branches to  the byte code at offset  8, otherwise a 1  is pushed onto
the operand stack  and the control flow branches  to the final return.
For ease of reading, the offsets of the branch instructions, which are
actually   relative,  are displayed  as  absolute  addresses in  these
examples.

If  recursion has to  continue, the  arguments for  the multiplication
(\texttt{n}  and \texttt{fac(n -  1)}) are  evaluated and  the results
pushed onto the operand stack.  After the multiplication operation has
been performed the function returns the computed value from the top of
the stack.

\subsubsection{Method readInt}

{\small \begin{verbatim}
0:  sipush        4711
3:  istore_0
4:  getstatic     java.lang.System.out Ljava/io/PrintStream;
7:  ldc           "Please enter a number> "
9:  invokevirtual java.io.PrintStream.print (Ljava/lang/String;)V
12: getstatic     Faculty.in Ljava/io/BufferedReader;
15: invokevirtual java.io.BufferedReader.readLine ()Ljava/lang/String;
18: invokestatic  java.lang.Integer.parseInt (Ljava/lang/String;)I
21: istore_0
22: goto          #44
25: astore_1
26: getstatic     java.lang.System.err Ljava/io/PrintStream;
29: aload_1
30: invokevirtual java.io.PrintStream.println (Ljava/lang/Object;)V
33: goto          #44
36: astore_1
37: getstatic     java.lang.System.err Ljava/io/PrintStream;
40: aload_1
41: invokevirtual java.io.PrintStream.println (Ljava/lang/Object;)V 
44: iload_0
45: ireturn

Exception handler(s) = 
From    To      Handler Type
4       22      25      java.io.IOException(6)
4       22      36      NumberFormatException(10)
\end{verbatim}}

First the local variable \texttt{n}  (in slot 0) is initialized to the
value  4711.   The  next  instruction, \texttt{getstatic},  loads  the
static  \texttt{System.out} field onto  the stack.   Then a  string is
loaded and  printed, a number   read from the  standard input and
assigned to \texttt{n}.

If    one   of   the    called   methods    (\texttt{readLine()}   and
\texttt{parseInt()}) throws  an exception, the  \jvm calls one  of the
declared exception  handlers, depending on the type  of the exception.
The \texttt{try}-clause  itself does not  produce any code,  it merely
defines the range in which  the following handlers are active.  In the
example  the specified  source  code area  maps  to a  byte code  area
ranging from offset 4 (inclusive)  to 22 (exclusive).  If no exception
has   occurred   (``normal''   execution   flow)   the   \texttt{goto}
instructions  branch behind  the  handler code.   There  the value  of
\texttt{n} is loaded and returned.

For  example the handler   for \texttt{java.io.IOException}  starts at
offset 25. It simply prints the error  and branches back to the normal
execution flow, i.e. as if no exception had occurred.

\section{The BCEL API}\label{sec:api}

The \jc API abstracts from  the concrete circumstances of the \jvm and
how  to  read and  write  binary Java  class  files.   The API  mainly
consists of three parts:

\begin{enumerate}
  
\item A package that contains classes that describe ``static''
  constraints of class files, i.e., reflect the class file format and
  is not intended for byte code modifications.  The classes may be
  used to read and write class files from or to a file.  This is
  useful especially for analyzing Java classes without having the
  source files at hand.  The main data structure is called
  \texttt{JavaClass} which contains methods, fields, etc..

\item A  package to dynamically generate  or modify \texttt{JavaClass}
objects.  It  may be  used  e.g. to  insert  analysis  code, to  strip
unnecessary  information from class  files, or  to implement  the code
generator back-end of a Java compiler.

\item Various code examples and  utilities like a class file viewer, a
tool  to convert class  files into  HTML, and  a converter  from class
files to the Jasmin assembly language \cite{jasmin}.
\end{enumerate}

\subsection{JavaClass}\label{sec:javaclass}

The  ``static''  component of  the  \jc  API  resides in  the  package
\path{de.fub.bytecode.classfile} and represents class files.  All of the
binary   components  and   data   structures  declared   in  the   JVM
specification  \cite{jvm} and described  in section  \ref{sec:jvm} are
mapped to classes.  Figure \ref{fig:umljc} shows an UML diagram of the
hierarchy of  classes of the  \jc API.  Figure \ref{fig:umlcp}  in the
appendix also  shows a  detailed diagram of  the \texttt{ConstantPool}
components.

\begin{figure}[htbp]
  \begin{center}
    \leavevmode 
    \epsfysize0.93\textheight
    \epsfbox{eps/javaclass.eps}
    \caption{UML diagram for the \jc API}\label{fig:umljc}
  \end{center}
\end{figure}

The  top-level data  structure  is \texttt{JavaClass},  which in  most
cases is created by  a \texttt{Class\-Par\-ser} object that is capable
of parsing  binary class files. A  \texttt{JavaClass} object basically
consists of  fields, methods, symbolic  references to the  super class
and to the implemented interfaces.

The  \cp serves  as some  kind of  central repository  and is  thus of
outstanding  importance  for  all  components.   \texttt{ConstantPool}
objects contain  an array of fixed size  of \texttt{Constant} entries,
which may be retrieved via the \texttt{getConstant()} method taking an
integer  index as argument.  Indexes to  the \cp  may be  contained in
instructions as well as in other components of a class file and in \cp
entries themselves.

Methods and  fields contain  a signature, symbolically  defining their
types.   Access  flags  like  \texttt{public static  final}  occur  in
several  places  and  are  encoded   by  an  integer  bit  mask,  e.g.
\texttt{public static final} matches to the Java expression

\begin{verbatim}
  int access_flags = ACC_PUBLIC | ACC_STATIC | ACC_FINAL;
\end{verbatim}

As mentioned in section \ref{sec:format} already, several components
may contain \emph{attribute} objects: classes, fields, methods, and
\texttt{Code} objects (introduced in section \ref{sec:code2}).  The
latter is an attribute itself that contains the actual byte code
array, the maximum stack size, the number of local variables, a table
of handled exceptions, and some optional debugging information coded
as \texttt{LineNumberTable} and \texttt{LocalVariableTable}
attributes. Attributes are in general specific to some data structure,
i.e. no two components share the same kind of attribute, though this
is not explicitly forbidden. In the figure the \texttt{Attribute}
classes are marked with the component they belong to.

\subsection{Class repository}

Using the provided \texttt{Repository} class, reading class files into
a \texttt{JavaClass} object is quite simple:

\begin{verbatim}
  JavaClass clazz = Repository.lookupClass("java.lang.String");
\end{verbatim}

The repository also contains methods providing the dynamic equivalent
of the \texttt{instanceof} operator, and other useful routines:

\begin{verbatim}
  if(Repository.instanceOf(clazz, super_class) {
    ...
  }
\end{verbatim}

\subsubsection{Accessing class file data}

Information within the class file components may be accessed like Java
Beans via intuitive set/get methods.  All of them also define a
\texttt{toString()} method so that implementing a simple class viewer
is very easy. In fact all of the examples used here have been produced
this way:

{\small \begin{verbatim}
  System.out.println(clazz);
  printCode(clazz.getMethods());
  ...
  public static void printCode(Method[] methods) {
    for(int i=0; i < methods.length; i++) {
      System.out.println(methods[i]);

      Code code = methods[i].getCode();
      if(code != null) // Non-abstract method
        System.out.println(code);
    }
  }
\end{verbatim}}

\subsubsection{Analyzing class data}

Last but not least, \jc supports the \emph{Visitor} design
pattern \cite{design},  so one can write visitor  objects to traverse
and analyze the contents of a class file. Included in the distribution
is a  class \texttt{JasminVisitor} that converts class  files into the
Jasmin assembler language \cite{jasmin}.

\subsection{ClassGen}\label{sec:cgapi}

This part of the API (package \path{ork.apache.bcel.generic}) supplies
an abstraction level for creating or transforming class files
dynamically.  It makes the static constraints of Java class files like
the hard-coded byte code addresses generic.  The generic \cpe, for
example, is implemented by the class \texttt{ConstantPoolGen} which
offers methods for adding different types of constants.  Accordingly,
\texttt{ClassGen} offers an interface to add methods, fields, and
attributes.  Figure \ref{fig:umlcg} gives an overview of this part of
the API.

\begin{figure}[htbp]
  \begin{center}
    \leavevmode
    \epsfysize0.93\textheight
    \epsfbox{eps/classgen.eps}
    \caption{UML diagram of the ClassGen API}\label{fig:umlcg}
  \end{center}
\end{figure}

\subsubsection{Types}

We abstract from the concrete details of the type signature syntax
(see \ref{sec:types}) by introducing the \texttt{Type} class, which is
used, for example, by methods to define their return and argument
types.  Concrete sub-classes are \texttt{BasicType},
\texttt{ObjectType}, and \texttt{ArrayType} which consists of the
element type and the number of dimensions. For commonly used types the
class offers some predefined constants.  For example the method
signature of the \texttt{main} method as shown in section
\ref{sec:types} is represented by:

\begin{verbatim}
  Type   return_type = Type.VOID;
  Type[] arg_types   = new Type[] { new ArrayType(Type.STRING, 1) };
\end{verbatim}

\texttt{Type} also contains methods to convert types into textual
signatures and vice versa. The sub-classes contain implementations of
the routines and constraints specified by the Java Language
Specification \cite{gosling}.

\subsubsection{Generic fields and methods}

Fields  are represented  by  \texttt{FieldGen} objects,  which may  be
freely  modified  by  the  user.   If  they  have  the  access  rights
\texttt{static final}, i.e. are constants  and of basic type, they may
optionally have an initializing value.

Generic  methods contain  methods  to add  exceptions  the method  may
throw,  local variables, and  exception handlers.  The latter  two are
represented by  user-configurable objects as  well.  Because exception
handlers  and   local  variables  contain  references   to  byte  code
addresses, they  also take the role of  an \emph{instruction targeter}
in   our  terminology.    Instruction  targeters   contain   a  method
\texttt{updateTarget()}    to   redirect    a    reference.    Generic
(non-abstract) methods refer  to \emph{instruction lists} that consist
of  instruction  objects.   References  to  byte  code  addresses  are
implemented by  handles to instruction  objects. This is  explained in
more detail in the following sections.

The maximum stack size needed by the method and the maximum number of
local variables used may be set manually or computed via the
\texttt{setMaxStack()} and \texttt{setMaxLocals()} methods
automatically.

\subsubsection{Instructions}

Modeling instructions as objects may look somewhat odd at first sight,
but in fact enables programmers to obtain a high-level view upon
control flow without handling details like concrete byte code offsets.
Instructions consist of a tag, i.e. an opcode, their length in bytes
and an offset (or index) within the byte code. Since many instructions
are immutable, the \texttt{InstructionConstants} interface offers
shareable predefined ``fly-weight'' constants to use.

Instructions are grouped via sub-classing, the type hierarchy of
instruction classes is illustrated by (incomplete) figure
\ref{fig:umlinstr} in the appendix.  The most important family of
instructions are the \emph{branch instructions}, e.g.  \texttt{goto},
that branch to targets somewhere within the byte code.  Obviously,
this makes them candidates for playing an \texttt{InstructionTargeter}
role, too. Instructions are further grouped by the interfaces they
implement, there are, e.g., \texttt{TypedInstruction}s that are
associated with a specific type like \texttt{ldc}, or
\texttt{ExceptionThrower} instructions that may raise exceptions when
executed.

All instructions can be traversed via \texttt{accept(Visitor v)} methods,
i.e., the Visitor design pattern. There is however some special trick
in these methods that allows to merge the handling of certain
instruction groups. The \texttt{accept()} do not only call the
corresponding \texttt{visit()} method, but call \texttt{visit()}
methods of their respective super classes and implemented interfaces
first, i.e. the most specific \texttt{visit()} call is last. Thus one
can group the handling of, say, all \texttt{BranchInstruction}s into
one single method.

For debugging purposes  it may even make sense  to ``invent'' your own
instructions. In a sophisticated code generator like the one used as a
backend of  the Barat framework  \cite{barat} one often has  to insert
temporary  \texttt{nop} (No  operation) instructions.   When examining
the produced  code it may  be very difficult  to track back  where the
\texttt{nop}  was actually  inserted.  One  could think  of  a derived
\texttt{nop2}   instruction   that   contains   additional   debugging
information. When  the instruction  list is dumped  to byte  code, the
extra data is simply dropped.

One  could also  think  of  new byte  code  instructions operating  on
complex numbers that  are replaced by normal byte  code upon load-time
or are recognized by a new JVM.

\subsubsection{Instruction lists}\label{sec:il}

An \emph{instruction list} is implemented by a list of
\emph{instruction handles} encapsulating instruction objects.
References to instructions in the list are thus not implemented by
direct pointers to instructions but by pointers to instruction
\emph{handles}. This makes appending, inserting and deleting areas of
code very simple. Since we use symbolic references, computation of
concrete byte code offsets does not need to occur until finalization,
i.e.  until the user has finished the process of generating or
transforming code.  We will use the term instruction handle and
instruction synonymously throughout the rest of the paper.
Instruction handles may contain additional user-defined data using the
\texttt{addAttribute()} method.

\paragraph{Appending.}
One can append instructions or  other instruction lists anywhere to an
existing  list.   The  instructions   are  appended  after  the  given
instruction  handle.   All append  methods  return  a new  instruction
handle which may  then be used as the target  of a branch instruction,
e.g..

{\small \begin{verbatim}
  InstructionList il = new InstructionList();
  ...
  GOTO g = new GOTO(null);
  il.append(g);
  ...
  InstructionHandle ih = il.append(InstructionConstants.ACONST_NULL);
  g.setTarget(ih);
\end{verbatim}}

\paragraph{Inserting.}
Instructions may be  inserted anywhere into an existing  list.  They are
inserted  before the  given  instruction handle.   All insert  methods
return a  new instruction handle which  may then be used  as the start
address of an exception handler, for example.

{\small \begin{verbatim}
  InstructionHandle start = il.insert(insertion_point,
                                      InstructionConstants.NOP);
  ...
  mg.addExceptionHandler(start, end, handler, "java.io.IOException");
\end{verbatim}}


\paragraph{Deleting.}
Deletion of instructions is also very straightforward; all instruction
handles and the contained instructions within a given range are
removed from the instruction list and disposed.  The \texttt{delete()}
method may however throw a \texttt{TargetLostException} when there are
instruction targeters still referencing one of the deleted
instructions.  The user is forced to handle such exceptions in a
\texttt{try-catch} block and redirect these references elsewhere. The
\emph{peep hole} optimizer described in section \ref{sec:nop} gives a
detailed example for this.

{\small \begin{verbatim}
  try {
    il.delete(first, last);
  } catch(TargetLostException e) {
    InstructionHandle[] targets = e.getTargets();
    for(int i=0; i < targets.length; i++) {
      InstructionTargeter[] targeters = targets[i].getTargeters();
      for(int j=0; j < targeters.length; j++)
         targeters[j].updateTarget(targets[i], new_target);
    }
  }
\end{verbatim}}

\paragraph{Finalizing.}
When the instruction list is ready to be dumped to pure byte code, all
symbolic references must be mapped to real byte code offsets.  This is
done by the \texttt{getByteCode()} method which is called by default
by \texttt{MethodGen.getMethod()}. Afterwards you should call
\texttt{dispose()} so that the instruction handles can be reused
internally. This helps to reduce memory usage.

\begin{verbatim}
  InstructionList il = new InstructionList();

  ClassGen  cg = new ClassGen("HelloWorld", "java.lang.Object",
                              "<generated>", ACC_PUBLIC | ACC_SUPER,
                              null);
  MethodGen mg = new MethodGen(ACC_STATIC | ACC_PUBLIC,
                               Type.VOID, new Type[] { 
                                 new ArrayType(Type.STRING, 1) 
                               }, new String[] { "argv" },
                               "main", "HelloWorld", il, cp);
  ...
  cg.addMethod(mg.getMethod());
  il.dispose(); // Reuse instruction handles of list
\end{verbatim}

\subsubsection{Code example revisited}

Using  instruction lists gives  us a  generic view  upon the  code: In
Figure  \ref{fig:il}   we  again  present   the  code  chunk   of  the
\texttt{readInt()}   method  of   the  faculty   example   in  section
\ref{sec:fac}:  The local  variables \texttt{n}  and  \texttt{e1} both
hold two references to  instructions, defining their scope.  There are
two \texttt{goto}s branching  to the \texttt{iload} at the  end of the
method. One of the exception handlers is displayed, too: it references
the start and the end of the \texttt{try} block and also the exception
handler code.

\begin{figure}[htbp]
  \begin{center}
    \leavevmode
    \epsfxsize\textwidth
    \epsfbox{eps/il.eps}
    \caption{Instruction list for \texttt{readInt()} method}
    \label{fig:il}
  \end{center}
\end{figure}

\subsubsection{Instruction factories}\label{sec:compound}

To simplify the creation of certain instructions the user can use the
supplied \texttt{InstructionFactory} class which offers a lot of
useful methods to create instructions from scratch. Alternatively, he
can also use \emph{compound instructions}: When producing byte code,
some patterns typically occur very frequently, for instance the
compilation of arithmetic or comparison expressions.  You certainly do
not want to rewrite the code that translates such expressions into
byte code in every place they may appear. In order to support this,
the \jc API includes a \emph{compound instruction} (an interface with
a single \texttt{getInstructionList()} method).  Instances of this
class may be used in any place where normal instructions would occur,
particularly in append operations.

\paragraph{Example: Pushing constants.}
Pushing constants  onto the  operand stack may  be coded  in different
ways.  As   explained  in   section  \ref{sec:code}  there   are  some
``short-cut'' instructions that can be  used to make the produced byte
code  more  compact.  The   smallest  instruction  to  push  a  single
\texttt{1} onto  the stack is  \texttt{iconst\_1}, other possibilities
are \texttt{bipush} (can be used to push values between -128 and 127),
\texttt{sipush}  (between  -32768 and  32767),  or \texttt{ldc}  (load
constant from \cpe).

Instead of repeatedly selecting  the most compact instruction in, say,
a switch, one can  use the compound \texttt{PUSH} instruction whenever
pushing a constant  number or string. It will  produce the appropriate
byte code instruction and insert entries into to \cp if necessary.

\begin{verbatim}
  il.append(new PUSH(cp, "Hello, world"));
  il.append(new PUSH(cp, 4711));
\end{verbatim}

\subsubsection{Code patterns using regular expressions}\label{sec:peephole}

When  transforming  code, for  instance  during  optimization or  when
inserting analysis  method calls,  one typically searches  for certain
patterns  of  code to  perform  the  transformation  at.  To  simplify
handling such situations \jc introduces a special feature: One can
search  for  given code  patterns  within  an  instruction list  using
\emph{regular  expressions}.   In  such expressions,  instructions  are
represented by symbolic names, e.g.  "\texttt{`IfInstruction'}".  Meta
characters  like  \verb|+|, \verb|*|,  and  \verb@(..|..)@ have  their
usual meanings. Thus, the expression

\begin{verbatim}
  "`NOP'+(`ILOAD__'|`ALOAD__')*"
\end{verbatim}

represents a  piece of  code consisting of  at least  one \texttt{NOP}
followed  by   a  possibly   empty  sequence  of   \texttt{ILOAD}  and
\texttt{ALOAD} instructions.

The  \texttt{search()} method  of class  \texttt{FindPattern}  gets an
instruction list and a regular  expression as arguments and returns an
array  describing   the  area  of   matched  instructions.  Additional
constraints to  the matching  area of instructions,  which can  not be
implemented via  regular expressions, may be  expressed via \emph{code
constraints}.

\subsubsection{Example: Optimizing boolean expressions.}

In Java, boolean  values are mapped to 1 and  to 0, respectively. Thus,
the simplest way to evaluate boolean expressions is to push a 1 or a 0
onto the operand stack depending on the truth value of the expression.
But this  way, the subsequent combination of  boolean expressions (with
\verb|&&|, e.g) yields  long chunks of code that push  lots of 1s and
0s onto the stack.

When the code has been finalized  these chunks can be optimized with a
\emph{peep  hole}  algorithm:  An  \texttt{IfInstruction}  (e.g.   the
comparison of two  integers: \texttt{if\_icmpeq}) that either produces
a  1  or  a 0  on  the  stack  and  is  followed by  an  \texttt{ifne}
instruction (branch  if stack value $\neq$  0) may be  replaced by the
\texttt{IfInstruction} with  its branch target replaced  by the target
of  the  \texttt{ifne}  instruction:

{\small \verbatimtabinput{bool.java}}

The  applied code  constraint  object ensures  that  the matched  code
really  corresponds to  the targeted  expression  pattern.  Subsequent
application of this algorithm removes all unnecessary stack operations
and  branch instructions from  the byte  code. If  any of  the deleted
instructions  is still  referenced by  an \texttt{InstructionTargeter}
object, the reference has to be updated in the \texttt{catch}-clause.

Code example \ref{sec:hello} gives a  verbose example of how to create
a class  file, while  example \ref{sec:nop} shows  how to  implement a
simple  peephole optimizer  and how  to deal  with \texttt{TargetLost}
exceptions.

\paragraph{Example application:}
The expression

\begin{verbatim}
  if((a == null) || (i < 2))
    System.out.println("Ooops");
\end{verbatim}

can be mapped to both of the chunks of byte code shown in figure
\ref{fig:code}. The left column represents the unoptimized code while
the right column displays the same code after an aggressively
optimizing peep hole algorithm has been applied:

\begin{figure}[hpt]
\begin{minipage}{0.49\textwidth}
  {\small \verbatimtabinput{unopt}} \vfil
\end{minipage}
\begin{minipage}{0.49\textwidth}
  {\small \verbatimtabinput{opt}} \vfil
\end{minipage}\label{fig:code}\caption{Optimizing boolean expressions}
\begin{center}

  
\end{center}
\end{figure}
\section{Application areas}\label{sec:application}

There are many possible application areas for \jc ranging from class
browsers, profilers, byte code optimizers, and compilers to
sophisticated run-time analysis tools and extensions to the Java
language \cite{agesen, myers}.

Compilers like the Barat compiler  \cite{barat} use \jc to implement a
byte code  generating back end.  Other possible  application areas are
the  static  analysis   of  byte code \cite{thies}  or  examining  the
run-time behavior  of classes by inserting calls  to profiling methods
into the  code. Further examples  are extending Java  with Eiffel-like
assertions  \cite{jawa}, automated delegation  \cite{classfilters}, or
with the concepts of ``Aspect-Oriented Programming'' \cite{aspect}.

\subsection{Class loaders}\label{sec:classloaders}

Class loaders  are responsible for  loading class files from  the file
system  or  other resources  and  passing the  byte  code  to the  \vm
\cite{classloader}.  A custom  \texttt{ClassLoader} object may be used
to  intercept the  standard procedure  of  loading a  class, i.e.  the
system class loader, and  perform some transformations before actually
passing the byte code to the JVM.

A  possible  scenario is  described  in figure  \ref{fig:classloader}:
During run-time the \vm requests a custom class loader to load a given
class.  But before  the JVM  actually sees  the byte  code,  the class
loader makes  a ``side-step'' and performs some  transformation to the
class. To  make sure that  the modified byte  code is still  valid and
does not violate any of the  JVM's rules it is checked by the verifier
before the JVM finally executes it.

\begin{figure}[ht]
  \begin{center}
    \leavevmode
    \epsfxsize\textwidth
    \epsfbox{eps/classloader.eps}
    \caption{Class loaders}\label{fig:classloader}
  \end{center}
\end{figure}

Using class loaders  is an elegant way of extending  the \jvm with new
features  without   actually  modifying  it.    This  concept  enables
developers to use \emph{load-time reflection} to implement their ideas
as opposed to  the static reflection supported by  the Java Reflection
API \cite{reflection}.  Load-time transformations supply the user with
a new  level of abstraction.   He is not  strictly tied to  the static
constraints of the  original authors of the classes  but may customize
the applications  with third-party code  in order to benefit  from new
features. Such  transformations may be executed on  demand and neither
interfere with other users, nor alter the original byte code. In fact,
class loaders may even create  classes \emph{ad hoc} without loading a
file at all.

\subsubsection{Example: Poor Man's Genericity}

The  ``Poor Man's  Genericity'' project  \cite{pmg} that  extends Java
with parameterized  classes, for  example, uses \jc  in two  places to
generate instances of  parameterized classes: During compile-time (the
standard  \texttt{javac} with  some slightly  changed classes)  and at
run-time  using  a  custom  class  loader.   The  compiler  puts  some
additional  type information into  class files  which is  evaluated at
load-time  by  the  class  loader.   The class  loader  performs  some
transformations on  the loaded  class and passes  them to the  VM. The
following  algorithm illustrates  how  the load  method  of the  class
loader   fulfills    the   request   for    a   parameterized   class,
e.g. \verb|Stack<String>|

\begin{enumerate}
\item  Search for  class  \texttt{Stack},  load it,  and  check for  a
certain class  attribute containing additional  type information. I.e.
the   attribute   defines   the    ``real''   name   of   the   class,
i.e. \verb|Stack<A>|.

\item  Replace  all  occurrences  and  references to  the  formal  type
\texttt{A}  with references  to the  actual type  \texttt{String}. For
example the method

\begin{verbatim}
  void push(A obj) { ... }
\end{verbatim}

becomes

\begin{verbatim}
  void push(String obj) { ... }
\end{verbatim}

\item Return the resulting class to the Virtual Machine.
\end{enumerate}

\bibliographystyle{alpha}\bibliography{manual}

\newpage\appendix

\pagestyle{empty}
\section{Code examples for the BCEL API}\label{sec:apicg}

\subsection{HelloWorldBuilder.java}
The following Java program reads a name from the standard input and
prints a friendly ``Hello''. Since the \texttt{readLine()} method may
throw an \texttt{IOException} it is enclosed by a \texttt{try-catch} block.

{\small \verbatimtabinput{HelloWorld.java}\label{sec:hello}}

\subsection{HelloWorldBuilder.java}

We will sketch  here how the above Java class can  be created from the
scratch  using the \jc API. For ease of reading we will
use textual signatures and not create them dynamically. For example,
the signature

\begin{verbatim}
  "(Ljava/lang/String;)Ljava/lang/StringBuffer;"
\end{verbatim}

would actually be created with

\begin{verbatim}
  Type.getMethodSignature(Type.STRINGBUFFER, new Type[] { Type.STRING });
\end{verbatim}

\subsubsection{Initialization:}

First we create an empty class and an instruction list:

{\small\begin{verbatim}
  ClassGen  cg = new ClassGen("HelloWorld", "java.lang.Object",
                              "<generated>", ACC_PUBLIC | ACC_SUPER,
                              null);
  ConstantPoolGen cp = cg.getConstantPool(); // cg creates constant pool
  InstructionList il = new InstructionList();
\end{verbatim}}

We then create the main method, supplying the method's name and the
symbolic type signature encoded with \texttt{Type} objects.

{\small\begin{verbatim}
  MethodGen  mg = new MethodGen(ACC_STATIC | ACC_PUBLIC,// access flags
                                Type.VOID,              // return type
                                new Type[] {            // argument types
                                  new ArrayType(Type.STRING, 1) },
                                new String[] { "argv" }, // arg names
                                "main", "HelloWorld",    // method, class
                                il, cp);
  InstructionFactory factory = new InstructionFactory(cg);
\end{verbatim}}

We define some often used types:

{\small\begin{verbatim}
  ObjectType i_stream = new ObjectType("java.io.InputStream");
  ObjectType p_stream = new ObjectType("java.io.PrintStream");
\end{verbatim}}

\subsubsection{Create variables \texttt{in} and \texttt{name}:}

We call the           constructors,              i.e.              execute
\texttt{BufferedReader(Input\-Stream\-Reader(System.in))}.  The  reference
to the \texttt{BufferedReader} object stays on top of the stack and is
stored in the newly allocated \texttt{in} variable.

{\small\begin{verbatim}
  il.append(factory.createNew("java.io.BufferedReader"));
  il.append(InstructionConstants.DUP); // Use predefined constant
  il.append(factory.createNew("java.io.InputStreamReader"));
  il.append(InstructionConstants.DUP);
  il.append(factory.createFieldAccess("java.lang.System", "in", i_stream,
                                      Constants.GETSTATIC));
  il.append(factory.createInvoke("java.io.InputStreamReader", "<init>",
                                 Type.VOID, new Type[] { i_stream },
                                 Constants.INVOKESPECIAL));
  il.append(factory.createInvoke("java.io.BufferedReader", "<init>", Type.VOID,
                                 new Type[] {new ObjectType("java.io.Reader")},
                                 Constants.INVOKESPECIAL));

  LocalVariableGen lg =
    mg.addLocalVariable("in",
                        new ObjectType("java.io.BufferedReader"), null, null);
  int in = lg.getIndex();
  lg.setStart(il.append(new ASTORE(in))); // `i' valid from here
\end{verbatim}}

Create local variable \texttt{name} and  initialize it to \texttt{null}.

{\small\begin{verbatim}
  lg = mg.addLocalVariable("name", Type.STRING, null, null);
  int name = lg.getIndex();
  il.append(InstructionConstants.ACONST_NULL);
  lg.setStart(il.append(new ASTORE(name))); // `name' valid from here
\end{verbatim}}

\subsubsection{Create try-catch block}

We remember  the start  of the  block, read a  line from  the standard
input and store it into the variable \texttt{name}.

{\small\begin{verbatim}
  InstructionHandle try_start =
    il.append(factory.createFieldAccess("java.lang.System", "out", p_stream,
                                        Constants.GETSTATIC));

  il.append(new PUSH(cp, "Please enter your name> "));
  il.append(factory.createInvoke("java.io.PrintStream", "print", Type.VOID, 
                                 new Type[] { Type.STRING },
                                 Constants.INVOKEVIRTUAL));
  il.append(new ALOAD(in));
  il.append(factory.createInvoke("java.io.BufferedReader", "readLine",
                                 Type.STRING, Type.NO_ARGS,
                                 Constants.INVOKEVIRTUAL));
  il.append(new ASTORE(name));
\end{verbatim}}

Upon  normal execution we  jump behind  exception handler,  the target
address is not known yet.

{\small\begin{verbatim}
  GOTO g = new GOTO(null);
  InstructionHandle try_end = il.append(g);
\end{verbatim}}

We add the exception handler which simply returns from the method.

{\small\begin{verbatim}
  InstructionHandle handler = il.append(InstructionConstants.RETURN);
  mg.addExceptionHandler(try_start, try_end, handler, "java.io.IOException");
\end{verbatim}}

``Normal'' code continues, now we can set the branch target of the GOTO.

{\small\begin{verbatim}
  InstructionHandle ih =
    il.append(factory.createFieldAccess("java.lang.System", "out", p_stream,
                                        Constants.GETSTATIC));
  g.setTarget(ih);
\end{verbatim}}

\subsubsection{Printing "Hello"}

String concatenation compiles to \texttt{StringBuffer} operations.

{\small\begin{verbatim}
  il.append(factory.createNew(Type.STRINGBUFFER));
  il.append(InstructionConstants.DUP);
  il.append(new PUSH(cp, "Hello, "));
  il.append(factory.createInvoke("java.lang.StringBuffer", "<init>",
                                 Type.VOID, new Type[] { Type.STRING },
                                 Constants.INVOKESPECIAL));
  il.append(new ALOAD(name));
  il.append(factory.createInvoke("java.lang.StringBuffer", "append",
                                 Type.STRINGBUFFER, new Type[] { Type.STRING },
                                 Constants.INVOKEVIRTUAL));
  il.append(factory.createInvoke("java.lang.StringBuffer", "toString",
                                 Type.STRING, Type.NO_ARGS,
                                 Constants.INVOKEVIRTUAL));
    
  il.append(factory.createInvoke("java.io.PrintStream", "println",
                                 Type.VOID, new Type[] { Type.STRING },
                                 Constants.INVOKEVIRTUAL));
  il.append(InstructionConstants.RETURN);
\end{verbatim}}

\subsubsection{Finalization}

Finally, we  have to  set  the stack  size,  which  normally would  be
computed on the fly and add a default constructor method to the class,
which is empty in this case.

{\small\begin{verbatim}
  mg.setMaxStack(5);
  cg.addMethod(mg.getMethod());
  il.dispose(); // Allow instruction handles to be reused
  cg.addEmptyConstructor(ACC_PUBLIC);
\end{verbatim}}

Last but not least we dump the \texttt{JavaClass} object to a file.

{\small\begin{verbatim}
  try {
    cg.getJavaClass().dump("HelloWorld.class");
  } catch(java.io.IOException e) { System.err.println(e); }
\end{verbatim}}

\subsection{Peephole.java}

This class implements a simple peephole optimizer that removes any NOP
instructions from the given class.

{\small\verbatimtabinput{Peephole.java}}\label{sec:nop}


\begin{figure}[htbp]
  \begin{center}
    \leavevmode
    \epsfysize0.93\textheight
    \epsfbox{eps/constantpool.eps}
    \caption{UML diagram for the ConstantPool API}\label{fig:umlcp}
  \end{center}
\end{figure}

\begin{figure}[ht]
  \begin{center}
    \leavevmode
    \epsfysize0.93\textheight
    \epsfbox{eps/instructions.eps}
    \caption{UML diagram for the Instruction API}\label{fig:umlinstr}
  \end{center}
\end{figure}



\end{document}
% LocalWords:  Freie Universit Institut Informatik dahm inf fu berlin de JOIE
% LocalWords:  JCF HelloWorld Jasmin Eiffel SourceFile classfile lang io ldc ar
% LocalWords:  PrintStream invokevirtual MethodRef ConstantClass Fieldref dup
% LocalWords:  iconst bipush iadd cmpeq jsr athrow iload istore iastore er ic
% LocalWords:  getfield putfield getstatic putstatic invokestatic newarray nop
% LocalWords:  anewarray checkcast instanceof lookupswitch tableswitch Barat VM
% LocalWords:  ConstantPool getConstant LineNumberTable LocalvariableTable ifne
% LocalWords:  invokesta invokespecial multianewarray ConstantPoolGen MethodGen
% LocalWords:  BasicType ObjectType ArrayType InstructionTarge getByteCode bool
% LocalWords:  BranchInstruction InstructionTargeter InstructionHandle regex
% LocalWords:  TargetLostException codeconstraint CompoundInstruction sipush
% LocalWords:  getInstructionList FindPattern CodeConstraint IfInstruction fac
% LocalWords:  TargetLost javac readLine IOException parseInt readInt CLASSPATH
% LocalWords:  NumberFormatException toString JasminVisitor getSignature JVM's
% LocalWords:  FieldGen updateTarget LGPL BCEL LocalVariableTable setMaxStack
% LocalWords:  setMaxLocals InstructionConstants TypedInstruction addAttribute
% LocalWords:  ExceptionThrower getMethod InstructionFactory ALOAD

